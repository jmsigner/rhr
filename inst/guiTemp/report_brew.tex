\documentclass[a4,oneside,garamond,11.5pt]{scrreprt}
\usepackage[utf8]{inputenc}
\usepackage[T1]{fontenc}
\usepackage{fixltx2e}
\usepackage{graphicx}
\usepackage{longtable}
\usepackage{float}
\usepackage{wrapfig}
\usepackage{rotating}
\usepackage[normalem]{ulem}
\usepackage{amsmath}
\usepackage{textcomp}
\usepackage{marvosym}
\usepackage[integrals]{wasysym}
\usepackage{amssymb}
\usepackage{hyperref}
\tolerance=1000
\usepackage{color}
\usepackage{listings}
\usepackage{amsmath}
\usepackage{lmodern}
\usepackage{tikz}
\usepackage{hyperref}
\usepackage{soul}
\usepackage{titlesec}
\usepackage{titletoc}
\usepackage{todonotes}
\usepackage{epigraph}
\usepackage{menukeys}
\usepackage[tikz]{bclogo}
\usepackage{tikz}
\usepackage{natbib}
\usepackage{booktabs}
\usepackage{mathtools}
\usepackage{placeins}
\usepackage[font=small,labelfont=bf]{caption}
\newcommand{\ra}[1]{\renewcommand{\arraystretch}{#1}}
\lstset{basicstyle=\ttfamily,
showstringspaces=false,
commentstyle=\color{red},
keywordstyle=\color{blue},
breaklines=true,
numberstyle=\tiny,
backgroundcolor=\color{black!5},
frame=rb,
framesep=2pt,
framerule=0.5pt,
fillcolor=\color{black!0}
}
\hypersetup{
colorlinks=false,
linkbordercolor={white},
citebordercolor={white}}

% Start renumbering sections within each part
\makeatletter
\@addtoreset{section}{part}   % reset section counter in every part
\makeatother

% Makes the TOC nicer, spaces are better allocated
\titlecontents{section}[2.3em]{}{\contentslabel{2.3em}}{\hspace*{-2.3em}}{\titlerule*[1pc]{.}\contentspage}
\author{R}
\date{\today}
\title{Home Range analysis with rhr}
\begin{document}

\maketitle
\tableofcontents


<<r setup, include=FALSE>>=
library(knitr)
opts_chunk$set(warning=FALSE, message=FALSE, echo=FALSE)
opts_knit$set(base.dir="<%=normalizePath(baseDir)%>")
options(width=65)
@


<%# ####################################
    General Information
  # #################################### -%>

\chapter{General Information}
\section{Relocations}
<<result='asis'>>=
bktbs <- xtable(relocTable)
hlns <- c(-1, 0, nrow(bktbs))
print(bktbs, booktabs=TRUE, hline.after=hlns, table.placement="!h", include.rownames=FALSE)
@

\section{Spatial Reference System}
The input SRS was: <%=epsgs$input %>. The output SRS was: <%=epsgs$output %>.

\section{Plot of relocation}

<<>>=
plot(data2[, c("lon", "lat")], col=adjustcolor("black", 0.1), pch=19, asp=1)
points(data3[ , c("lon", "lat")], col="red")
@


<%# ####################################
    Estimators
  # #################################### -%>

<%# -------------------------- START CYCLE OVER ALL ESTIMATORS -%>
<% for (step in methodLookup$short) { %>

\chapter{<%=methodLookup[methodLookup$short == step, 'long']%>}

<% if (step %in% names(res$est)) { %>

<% if (!is.null(res$est[[step]]$parameters)) { %>
\section{Parameter values used}
<<results="asis">>=
bktbs <- xtable(readRDS("<%=normalizePath(res$est[[step]]$parameters)%>"))
hlns <- c(-1, 0, nrow(bktbs))
print(bktbs, booktabs=TRUE, hline.after=hlns, table.placement="!htb", include.rownames=FALSE)
@
\FloatBarrier

<%}%>


<%# -------------------------- START CYCLE OVER ALL ANIMALS -%>
<% for (animal in unique(dat$id)) { %>

\section{Results for <%=animal%>}


<%# -------------------------- START CYCLE OVER ALL SCENARIOS OF AN ANIMAL -%>
<% for (scenario in 1:length(res$est[[step]]$res[[animal]])) { %>

<% if (length(res$est[[step]]$res[[animal]]) > 1) { %>
\subsection{ Scenario: <%=res$est[[step]]$scenarios$name[scenario]%>}
<%}%>

<%# ========================== Check if I can continue, that means no error occured during calculation %>
<% if (is(readRDS(normalizePath(res$est[[step]]$res[[animal]][[scenario]]$rds)), "error")) { %>
Terminated with error: <%=readRDS(normalizePath(res$est[[step]]$res[[animal]][[scenario]])$rds)$message %>
<% } else { %>

<%# -------------------------- plots -%>
<% if (!is.null(res$est[[step]]$res[[animal]][[scenario]]$plots)) { %>

<% for (plt in seq_along(res$est[[step]]$res[[animal]][[scenario]]$plots)) { %>

\subsubsection{ <%=res$est[[step]]$res[[animal]][[scenario]]$plots[[plt]]$name%>}

\begin{figure}[!htb]
\centering
\includegraphics[width=0.7\textwidth]{<%=normalizePath(res$est[[step]]$res[[animal]][[scenario]]$plots[[plt]]$plotPDF) %>}
\end{figure}
\FloatBarrier
  
<% if (!is.null(res$est[[step]]$res[[animal]][[scenario]]$plots[[plt]]$legend)) { %>
<%=res$est[[step]]$res[[animal]][[scenario]]$plots[[plt]]$legend %>
<%}%>

<% } %>
<% } %>

<%# -------------------------- table -%>
<% if (!is.null(res$est[[step]]$res[[animal]][[scenario]]$tables)) { %>

<% for (tbl in seq_along(res$est[[step]]$res[[animal]][[scenario]]$tables)) { %>

\subsubsection{ <%=res$est[[step]]$res[[animal]][[scenario]]$tables[[tbl]]$name%>}

<<results="asis">>=
bktbs <- xtable(readRDS("<%=normalizePath(res$est[[step]]$res[[animal]][[scenario]]$tables[[tbl]]$path)%>"))
hlns <- c(-1, 0, nrow(bktbs))
print(bktbs, booktabs=TRUE, hline.after=hlns, table.placement="!h", include.rownames=FALSE)
@

\FloatBarrier
<% if (!is.null(res$est[[step]]$res[[animal]][[scenario]]$table[[tbl]]$legend)) { %>
<%=res$est[[step]]$res[[animal]][[scenario]]$tables[[tbl]]$legend %>
<%}%>

<% } %>
<% } %>

<%# -------------------------- message -%>
<% if (!is.null(res$est[[step]]$res[[animal]][[scenario]]$messages)) { %>

<% for (msg in seq_along(res$est[[step]]$res[[animal]][[scenario]]$messages)) { %>

\subsubsection{ <%=res$est[[step]]$res[[animal]][[scenario]]$messages[[msg]]$name%>}

<<r, results="asis">>=
cat(readRDS("<%=res$est[[step]]$res[[animal]][[scenario]]$messages[[msg]]$message%>"))
@

<% } %>
<% } %>


<% } %>

<%# END CYCLE OVER ALL SCENARIOS %>
<% } %>

<%# END CYCLE OVER ALL ANIMALS %>

<% } %>

\newpage
<%} else { %>
Method was not requested
<% } %>


<%}%>
<%# END CYCLE OVER ALL METHODS %>

\chapter{Summary}
\section{Timing}
The analysis started at and took 123 mins.

\section{Session Info}
For the sake of reproducibility, you should always include your session info. This shows which version of R and different packages you were using. 

<<>>=
sessionInfo()
@

\section{About}
\subsection{Licence}
This report was generated as part of the rhr package for R. The package is licened uner GPLv3. This means you are free to use and modify the package. But the packages comes with NO WARRANTY whatsoever. 

\subsection{Help and more information}
Request for help should be directed to the corresponding user group (rhr-discussion on google groups). Please contact us (jsigner@gwdg.de) for bug reports. 

\subsection{Printing}
We hope that reports generated with this package will be useful make analysis easier and research more reproducible. By design quiet large pdf can be generated with this package. Please think twice before printing endless pages of preliminary results. 


\end{document}
